\section{Procesamiento del lenguaje natural}

El t\'ermino "Procesamiento del lenguaje natural" (PLN) hace referencia a las t\'ecnicas de tratamiento del lenguaje y su aplicaci\'on en las diversas \'areas por medio de m\'etodos computacionales.  \cite{elprofesionaldelainformacion}

El PLN es un \'area de investigaci\'on en continuo desarrollo, es aplicado en la actualidad para la realizaci\'on de diversas actividades como son: traducci\'on autom\'atica, sistemas de recuperaci\'on de informaci\'on, elaboraci\'on autom\'atica de res\'umenes, etc\'etera.  \cite{elprofesionaldelainformacion}

Si bien es cierto que la evoluci\'on de dicha \'area la posiciona para liderar una nueva dimensi\'on en las aplicaciones inform\'aticas del futuro, la complejidad impl\'icita en el tratamiento del lenguaje tiene sus limitaciones en los resultados.

Podemos definir el PLN como el reconocimiento y utilizaci\'on de la informaci\'on expresada en lenguaje humano a trav\'es del uso de sistemas inform\'aticos. En su estudio intervienen diferentes disciplinas: la ling\"u\'istica, ingenier\'ia inform\'atica, filosof\'ia, matem\'aticas y psicolog\'ia. \cite{elprofesionaldelainformacion}

Es importante el estudio del lenguaje para determinar el uso que implica \'este en diferentes tareas y de esta manera poder moldear el conocimiento de una manera adecuada. \cite{elprofesionaldelainformacion}

Es necesario tener en cuenta dos puntos: El primero tener en cuenta el problema de representaci\'on ling\"u\'istica para el cual se desea aplicar dicha t\'ecnica. Y en segunda el problema de tratamiento mediante recursos inform\'aticos.
\pagebreak

 %\cite{elprofesionaldelainformacion}

%\begin{thebibliography}{99}

%\bibitem{elprofesionaldelainformacion} Sosa, Eduardo. "Procesamiento del lenguaje natural: revisi\'on del estado actual, bases te\'oricas y aplicaciones (Parte I)". El Profesional de la Informaci\'on. N.p., Jan. 1997. Web. 6 Apr. 2014. 
%\end{thebibliography}