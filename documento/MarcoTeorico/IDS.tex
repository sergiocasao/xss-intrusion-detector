\section{Detección de Intrusos}

\subsection{Definición}

Detección de intrusos es el proceso de monitorear los eventos que ocurren en un sistema de cómputo o red y analizarlos por firmas o posibles incidentes que son violaciones o amenazas inminentes de violación de políticas de seguridad, políticas de uso aceptable o políticas de seguridad estándar. La prevención de intrusos es el proceso de realizar detección de intrusos e intentar detener el posible incidente detectado. Los sistemas de detección y prevención de intrusos (IDPS por sus siglas en inglés) son principalmente enfocados en identificar posibles incidentes, registrar información de ellos, intentar detenerlos y reportarlos al administrador de seguridad. La intrusión detectada puede ser efectuada desde el exterior y/o interior de una red o segmento que derive de ella. Algunas organizaciones usan los IDPSs con otros propósitos, ya sea para identificar problemas con sus políticas de seguridad, documentar las amenazas existentes o para disuadir a los individuos de violaciones de las políticas de seguridad \cite{nist94}. \\

\subsection{Taxonomía de incidentes de seguridad}

\subsection{Modo de operación}

Los IDSs están integrados por diversos módulos que trabajan en conjunto con funciones específicas la recolección de datos y el análisis de los mismos efectuados por un sistema, también la generación de alertas y una posible respuesta del tipo pasivo, activo o pro-activo. El registro de los resultados y datos que se obtiene se almacenan en bitácoras. El motor de detección de los IDSs emplea diversas formas de análisis dependiendo de su objetivo, algunas de estas formas son: estadísticos, de Inteligencia Artificial, Sistema Inmune, Machine Learning, como es este caso, entre otras formas. La operación de estos sistemas se puede contemplar en un ambiente aislado o con la interacción de otros controles de seguridad. Este último punto es muy importante tener en consideración, ya que dependiendo de dicha operación, afecta la forma en que opera el IDS y su configuración. \\

Los IDS pueden ser desarrolladas tanto en hardware como en software, cada uno con sus respectivas ventajas y desventajas. El desarrollo en hardware es un equipo de cómputo que debe ser implementado la arquitectura de una red, lo que implica una instalación y configuración por personas especializadas, la principal ventaja de éste desarrollo consiste en una independencia de un equipo de cómputo, sino de la robustez de los circuitos integrados y las partes que lo constituyen. El segundo desarrollo, de software, se implementa para una operación dentro de un equipo de cómputo dedicado, el cuál dependerá totalmente del sistema operativo en el equipo, implicando esto una configuración de varios componentes del equipo, así como las propias exigencias que se requieran del equipo de cómputo; memoria, almacenamiento, velocidad de procesamiento, etc.). Su ventaja radica en que pueden ser implementados directamente sobre la aplicación o sistema a monitorear \cite{dieciseis}. \\

\subsection{Justificación de los Sistemas de Detección de Intrusos}

Los sistemas de detección de intrusos (IDS por sus siglas en inglés) es un control de seguridad que debe ser implementado junto con otros controles de seguridad para fortalecer y\/o complicar la acción de una contra-parte, como es el caso de un \textit{cortafuegos}\footnote{Un cortafuegos o \textit{firewall}, por su nombre en inglés, son dispositivos o programas que controlan el flujo del tráfico de red entre redes o computadoras que emplean diferentes posturas de seguridad\cite{nist41}.}. La implementación de estos dos controles de seguridad son comúnmente empleados ya que el trabajo del cortafuegos el filtrar el tráfico de la red con base a un análisis de filtrado de paquetes o un filtrado de estado. Así, los IDSs reciben tráfico filtrado y reconocido para su análisis de acuerdo a diversos criterios dependiendo de la taxonomía implementada (que se definirá después). \\

Existen hoy en día entidades que emplean IDS dentro de los cortafuegos, ya que son la primera línea de seguridad defensiva de una entidad, con el objetivo de complementar su sistema de filtrado, y así ser más eficiente y oportuno durante un ataque o intento de intrusión. Pero dicha implementación no es que sea mejor que una de forma separada entre controles, más bien radica en otros factores como la cantidad de dispositivos existentes en la entidad que lo va a implementar, la cantidad de información que va a procesar y principalmente los recursos monetarios disponibles de la entidad, haciendo mención también que al juntar estos controles, el tiempo de procesamiento de los datos dependería mucho del hardware del dispositivo, así haciendo dependiente el flujo sin retardos de la red al dispositivo. También hay que tener en consideración que si la implementación de diferentes controles de seguridad se hace en un mismo dispositivo, existe un mayor riesgo de que si el dispositivo falla o es comprometido, la entidad pueda sufrir un ataque o una intrusión. \\

