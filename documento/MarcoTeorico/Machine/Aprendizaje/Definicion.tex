\subsection{Concepto}

En el concepto de aprendizaje una función objetivo es representada como un conjunto de restricciones sobre atributos. La hipótesis de un espacio \textit{H} consiste en el enrejado de posibles conjunciones de restricciones de atributo para un dominio de problema dado. Una estrategia de búsqueda de menor compromiso es adoptada para eliminar hipótesis en \textit{H} que no son consistentes con el entrenamiento establecido \textit{D}. Esto resultará en una estructura llamada el espacio de versión, el subconjunto de hipótesis que son consistentes con los datos de entrenamiento. El algoritmo, llamado la eliminación de candidatos, utiliza los operadaores de generalización y especialización para producir el espacio de version con consideración a \textit{H} y \textit{D}. Esto se basa en un sesgo de lenguaje (o restricción) que establece que la función objetivo está contenida en \textit{H}. Esto es un método de aprendizaje ansioso y supervisado \cite{defmach}.\\