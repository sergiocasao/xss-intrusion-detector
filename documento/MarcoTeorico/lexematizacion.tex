\section{Lematizaci\'on}
Uno de los procesos fundamentales para tratar los textos ling\"u\'isticos es la lematizaci\'on, que consiste en asignar una forma representativa a distintas formas concretas variables: formas conjugadas del verbo, cambios seg\'un el g\'enero y n\'umero de adjetivos y sustantivos, etc. Es necesario a la hora de realizar unos estudios estad\'isticos, redactar un vocabulario o diccionario, intentar una b\'usqueda de informaci\'on por palabras clave y analizar la combinaci\'on de elementos, entre otros objetivos de investigaci\'on.

Este proceso puede involucrar tareas complejas como entender el contexto de una categor\'ia gramatical de una palabra dentro de una oraci\'on. 

En muchos idiomas, las palabras aparecen flexionadas de difentes maneras. Por ejemplo en espa\~nol el verbo 'caminar' puede aparecer como 'camino', 'camin\'o', 'camin\'a' 'caminando'. La forma base 'caminar', que puede ser buscada y encontrada en un diccionario, es llamada lema de la palabra. La combinaci\'on de la foma base con la flexi\'on es llamada lexema de la palabra.


La importancia de la lematizaci\'on radica en el hecho que, para acceso por contenido a bases de datos textuales, permite superar las limitaciones de una b\'usqueda simple de strings, haciendo que relaciones ocultas por la variabilidad morfol\'ogica de las palabras queden manifiestas. La lematizaci\'on mejora por lo tanto el recubrimiento (recall) aunque pueda ser a expensas de la precisi\'on cuando diferentes conjugaciones morfol\'ogicas de una misma raiz est\'an asociadas a conceptos distintos.

La lematizaci\'on est\'a muy relacionada con el etiquetado autom\'atico de textos (POS tagging), que consiste en atribuir a cada palabra su categor\'ia gramatical, ya que la categor\'ia puede determinarse por las flexiones o derivaciones (ej: en castellano -ar indica un infinitivo, -ado un participio pasado masculino singular, etc.). Muchos esquemas de procesamiento de textos, aplicados a lenguas flexivas europeas, plantean un etiquetado autom\'atico previo a la lematizaci\'on, de manera que al lematizar se cuente con la informaci\'on de la categor\'ia gramatical de las palabras. Sin embargo, la atribuci\'on de etiquetas correctas depende en general de una lematizaci\'on impl\'icita basada en un an\'alisis de sufijos y prefijos, lo que permite una primera predicci\'on que se corrige, en una segunda etapa, en funci\'on del contexto immediato de la palabra analizada (Brill). Esta manera de proceder presenta algunos problemas: (i) requiere de un corpus manualmente etiquetado de gran dimensi\'on para derivar reglas de etiquetado autom\'atico adecuadas, (ii) no aprovecha la existencia de paradigmas de conjugaci\'on o derivaci\'on, (iii) s\'olo considera ra\'ices libres. \cite{lem}