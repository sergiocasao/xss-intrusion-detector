\section{Lenguaje natural}
El lenguaje se considera como un mecanismo que nos permite hablar y entender. Los lenguajes naturales, es decir, el ingl\'es, el franc\'es, el espa\~nol, etc. son una herramienta genuina para la comunicaci\'on entre los seres humanos, ya sea en forma oral o escrita.
Actualmente, el avance tecnol\'ogico en los medios de comunicaci\'on impresos y electr\'onicos nos permite obtener grandes vol\'umenes de informaci\'on en forma escrita. La mayor\'ia de esta informaci\'on se presenta en forma de textos en lenguajes naturales. Toda esa informaci\'on contenida en los textos es muy importante ya que permite analizar, comparar, entender el entorno en el que vive el ser humano.
Sin embargo, se presentan dificultades por la imposibilidad humana de manejar esa enorme cantidad de textos. Entre las herramientas que ayudan en las tareas diarias, la computadora es, hoy en d\'ia, una herramienta indispensable para el procesamiento de grandes vol\'umenes de datos. Pero todav\'ia no se logra que una m\'aquina al capturar una colecci\'on de textos los comprenda suficientemente bien; por ejemplo, para que pueda aconsejar qu\'e hacer en determinado momento bas\'andose en toda la informaci\'on proporcionada, para que pueda responder a preguntas acerca de los temas contenidos en esa informaci\'on pero no expl\'icitamente descritos, o para que pueda elaborar un resumen de la informaci\'on.
Para lograr esta enorme tarea de procesamiento de lenguaje natural por computadora, analizando oraci\'on por oraci\'on para obtener el sentido de los textos, es necesario conocer las reglas y los principios bajo los cuales funciona el lenguaje, a fin de reproducirlos y adecuarlos a la computadora, incluyendo posteriormente el procesamiento de lenguaje natural en el proceso general del conocimiento y el razonamiento.
El estudio del lenguaje, est\'a relacionado con diversas disciplinas. De entre ellas, la Ling\"u\'istica General es el estudio te\'orico que se ocupa de los m\'etodos de investigaci\'on y de las cuestiones comunes a las diversas lenguas. Esta disciplina a su vez comprende una multitud de aspectos (temporales, metodol\'ogicos, sociales, culturales, de aprendizaje, etc.). Los aspectos metodol\'ogicos y de aplicaci\'on brindan los principios y las reglas necesarios en el procesamiento de textos.
Los principios y las reglas de la ling\"u\'istica general, aunados a los m\'etodos de la computaci\'on forman la Ling\"u\'istica Computacional. Esta es la \'area dentro de la cu\'al se han desarrollado y discutido muchos formalismos adecuados para la computadora a fin de reproducir el funcionamiento del lenguaje con la finalidad de extraer sentido a partir de textos y viceversa, transformando los conceptos de sentidos espec\'ificos a los correspondientes textos correctos. \cite{cic}