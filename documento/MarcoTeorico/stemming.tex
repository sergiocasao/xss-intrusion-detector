\section{\textit{Stemming}}
Frecuentemente se usa una palabra para realizar alguna consulta, pero \'unicamente una variable de esa palabra est\'a presente en el texto ya sea en singular, plural, gerundio, etc\'etera.  El problema puede ser resulto con la sustituci\'on de una palabra por todas sus formas.

En la mayor\'ia de los casos, las variantes morfol\'ogicas de las palabras tienen interpretaciones sem\'anticas similares y se pueden considerar como equivalentes. Por esta raz\'on, se han desarrollado algoritmos de derivaci\'on, o analizadores ling\"u\'isticos, que tratan de reducir una palabra a su \textit{stem} o ra\'iz. Por lo tanto, los t\'erminos clave de una consulta o documento est\'an representados por las ra\'ices en lugar de las palabras originales. Esto no s\'olo significa que las diferentes variantes de un t\'ermino pueden confundir a una sola forma representativa, sino que tambi\'en reduce el tama\~no del diccionario, es decir, el n\'umero de t\'erminos distintos necesario para la representaci\'on de un conjunto de documentos. A peque\~nos resultados del tama\~no del diccionario en un ahorro de espacio de almacenamiento y tiempo de procesamiento.

El \textit{Stemming} es un m\'etodo utilizado para reducir una palabra a su ra\'iz. \cite{stemm}

\pagebreak