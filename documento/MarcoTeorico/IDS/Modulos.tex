\subsection{Componentes elementales de un IDS}

Con las propuestas descritas de las arquitecturas y modelos de las buenas prácticas para el desarrollo de IDSs, y con base a las especificaciones que nos proporciona el documento NIST 800-94, se puede generar un esquema genérico de los componentes elementales que deben ser implementados en todo IDS. Dichos componentes elementales que deben estar en un IDS, son: \\

\begin{itemize}
	
	\item \textbf{Sensor o Agente.} Los sensores y los agentes moniorean y analizan la actividad. El término \textit{sensor} es mayormente usado para IDPS que monitorean redes, incluyendo las basadas en red, no guiadas y tecnologías de análisis de comportamiento de red. El término \textit{agente} es usualmente utilizado para referirse a las tecnologías usadas para un análisis basado en host en un IDPS.
	
	\item \textbf{Servidor Administrador.} Este servidor administrador es el que recibe información de los agentes o sensores y administrarlos.Existen algunos servidores que hacen la función de análisis sobre la información enviada por los agentes o sensores e identificar eventos que por sí solos, los agentes o sensores no pueden identificarlos.
	
	\item \textbf{Servidor de Base de Datos.} Es un servidor en donde se va a almacenar toda la información registrada por los eventos o agentes, o también por el administrador. Esta información almacenada, no necesariamente será de eventos registrados, también puede ser del estado del sistema o de su comportamiento en su ejecución. Este componente es importante para los administradores ya que con la información contenida en este componente, se pueden identificar eventos no alertados o también conocidos como falso negativo para su posterior análisis o una proposición de modificaciones a los sensores o agentes.
	
	\item \textbf{Consola.} Es un programa que brinda una interfaz para el IDS para los usuarios y los administradores de éste. La consola es regularmente instalada de manera aislada en un equipo de cómputo común, como una computadora de escritorio o una computadora personal. Las consolas pueden ser usadas tanto para la administración de los agentes o sensores, como para un monitorear y analizar.
	
	\item \textbf{Respuesta.} El propósito de éste componente es proporcionar respuestas tanto Activas, Pasivas o Pro-activas. Las respuestas activas son aquellas en las que al momento de detectar una intrusión, se toman decisiones pre-configuradas en el sistema, como un bloqueo de direcciones IP, finalización de una conexión e incluso, la modificación de reglas de un control de seguridad. La pasivas son aquellas en las que se espera la intervención de un administrador para tomar las acciones necesarias sobre el evento ocurrido. Las respuestas pro-activas emplean el concepto del cómputo proactivo, es decir, la anticipación de una acción basada en lo que percibe del medio físico que se le va presentando. Cabe mencionar que independientemente de la respuesta a implementar elegida, las tres respuestas envían notificaciones/alertas de eventos en curso o pasados.
	
\end{itemize}