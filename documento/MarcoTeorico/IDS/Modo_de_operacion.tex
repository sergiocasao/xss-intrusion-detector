\subsection{Modo de operación}

Los IDSs están integrados por diversos módulos que trabajan en conjunto con funciones específicas la recolección de datos y el análisis de los mismos efectuados por un sistema, también la generación de alertas y una posible respuesta del tipo pasivo, activo o pro-activo. El registro de los resultados y datos que se obtiene se almacenan en bitácoras. El motor de detección de los IDSs emplea diversas formas de análisis dependiendo de su objetivo, algunas de estas formas son: estadísticos, de Inteligencia Artificial, Sistema Inmune, Machine Learning, como es este caso, entre otras formas. La operación de estos sistemas se puede contemplar en un ambiente aislado o con la interacción de otros controles de seguridad. Este último punto es muy importante tener en consideración, ya que dependiendo de dicha operación, afecta la forma en que opera el IDS y su configuración. 
\\

Los IDS pueden ser desarrolladas tanto en hardware como en software, cada uno con sus respectivas ventajas y desventajas. El desarrollo en hardware es un equipo de cómputo que debe ser implementado la arquitectura de una red, lo que implica una instalación y configuración por personas especializadas, la principal ventaja de éste desarrollo consiste en una independencia de un equipo de cómputo, sino de la robustez de los circuitos integrados y las partes que lo constituyen. El segundo desarrollo, de software, se implementa para una operación dentro de un equipo de cómputo dedicado, el cuál dependerá totalmente del sistema operativo en el equipo, implicando esto una configuración de varios componentes del equipo, así como las propias exigencias que se requieran del equipo de cómputo; memoria, almacenamiento, velocidad de procesamiento, etc.). Su ventaja radica en que pueden ser implementados directamente sobre la aplicación o sistema a monitorear \cite{dieciseis}. \\