\subsection{Justificación de los Sistemas de Detección de Intrusos}

Los sistemas de detección de intrusos (IDS por sus siglas en inglés) es un control de seguridad que debe ser implementado junto con otros controles de seguridad para fortalecer y\/o complicar la acción de una contra-parte, como es el caso de un \textit{cortafuegos}\footnote{Un cortafuegos o \textit{firewall}, por su nombre en inglés, son dispositivos o programas que controlan el flujo del tráfico de red entre redes o computadoras que emplean diferentes posturas de seguridad\cite{nist41}.}. La implementación de estos dos controles de seguridad son comúnmente empleados ya que el trabajo del cortafuegos el filtrar el tráfico de la red con base a un análisis de filtrado de paquetes o un filtrado de estado. Así, los IDSs reciben tráfico filtrado y reconocido para su análisis de acuerdo a diversos criterios dependiendo de la taxonomía implementada (que se definirá después). 
\\

Existen hoy en día entidades que emplean IDS dentro de los cortafuegos, ya que son la primera línea de seguridad defensiva de una entidad, con el objetivo de complementar su sistema de filtrado, y así ser más eficiente y oportuno durante un ataque o intento de intrusión. Pero dicha implementación no es que sea mejor que una de forma separada entre controles, más bien radica en otros factores como la cantidad de dispositivos existentes en la entidad que lo va a implementar, la cantidad de información que va a procesar y principalmente los recursos monetarios disponibles de la entidad, haciendo mención también que al juntar estos controles, el tiempo de procesamiento de los datos dependería mucho del hardware del dispositivo, así haciendo dependiente el flujo sin retardos de la red al dispositivo. También hay que tener en consideración que si la implementación de diferentes controles de seguridad se hace en un mismo dispositivo, existe un mayor riesgo de que si el dispositivo falla o es comprometido, la entidad pueda sufrir un ataque o una intrusión. 
\\