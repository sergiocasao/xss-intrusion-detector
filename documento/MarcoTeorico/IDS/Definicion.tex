\section{Detección de Intrusos}

\subsection{Definición}

Detección de intrusos es el proceso de monitorear los eventos que ocurren en un sistema de cómputo o red y analizarlos por firmas o posibles incidentes que son violaciones o amenazas inminentes de violación de políticas de seguridad, políticas de uso aceptable o políticas de seguridad estándar. La prevención de intrusos es el proceso de realizar detección de intrusos e intentar detener el posible incidente detectado. Los sistemas de detección y prevención de intrusos (IDPS por sus siglas en inglés) son principalmente enfocados en identificar posibles incidentes, registrar información de ellos, intentar detenerlos y reportarlos al administrador de seguridad. La intrusión detectada puede ser efectuada desde el exterior y/o interior de una red o segmento que derive de ella. Algunas organizaciones usan los IDPSs con otros propósitos, ya sea para identificar problemas con sus políticas de seguridad, documentar las amenazas existentes o para disuadir a los individuos de violaciones de las políticas de seguridad \cite{nist94}. \\