\section{An\'alisis sem\'antico}

En muchas aplicaciones del PLN los objetivos del an\'alisis apuntan hacia el procesamiento del significado. En los \'ultimos a\~nos las t\'ecnicas de procesamiento sint\'actico han experimentado avances significativos, resolviendo los problemas fundamentales.
Sin embargo, las t\'ecnicas de representaci\'on del significado no han obtenido los resultados deseados, y numerosas cuestiones contin\'uan sin encontrar soluciones satisfactorias.
Definir qu\'e es el significado no es una tarea sencilla, y puede dar lugar a diversas interpretaciones. A efectos funcionales, para facilitar el procesamiento, la modularidad es una de las propiedades m\'as deseables. Haciendo uso de esta concepci\'on modular es posible distinguir entre significado independiente y significado dependiente del contexto.
El primero, tratado por la sem\'antica, hace referencia al significado que las palabras tienen por s\'i mismas sin considerar el significado adquirido seg\'un el uso en una determinada circunstancia. La sem\'antica, por tanto, hace referencia a las condiciones de verdad de la frase, ignorando la influencia del contexto o las intenciones del hablante. Por otra parte, el componente significativo de una frase asociado a las circunstancias en que \'esta se da, es estudiado por la pragm\'atica y conocido como significado dependiente del contexto.
Atendiendo al desarrollo en el proceso de interpretaci\'on sem\'antica, es posible optar entre m\'ultiples pautas para su organizaci\'on, tal como se determinan en los siguientes p\'arrafos.
En referencia a la estructura sem\'antica que se va a generar, puede interesarnos que exista una simetr\'ia respecto a la estructura sint\'actica, o por el contrario que no se d\'e tal correspondencia entre ellas. En el primer caso, a partir del \'arbol generado por el an\'alisis sint\'actico se genera una estructura arb\'orea con las mismas caracter\'isticas, sobre la cual se realizar\'a el an\'alisis sem\'antico. En el segundo caso, en la estructura generada por la sintaxis se produce un curso de transformaciones sobre las cuales se genera la representaci\'on sem\'antica.
Cada una de las dos opciones anteriores puede implementarse de forma secuencial o paralela. En la interpretaci\'on secuencial, despu\'es de haber finalizado la fase de an\'alisis sint\'actico, se genera el an\'alisis sem\'antico. En cambio, desde un procedimiento en paralelo, el proceso de an\'alisis sem\'antico no necesita esperar a que el analizador sint\'actico haya acabado toda su tarea, sino que puede ir realizando el an\'alisis de cada constituyente cuando \'este ha sido tratado en el proceso sint\'actico.
Finalmente en combinaci\'on con cada una de las opciones anteriores, podemos escoger un modelo en el que exista una correspondencia entre reglas sint\'acticas y sem\'anticas o, contrariamente, podemos optar por un modelo que no cumpla tal requisito. En caso afirmativo, para cada regla sint\'actica existir\'a una regla sem\'antica correspondiente.
El significado es representado por formalismos conocidos por el nombre de knowledge representation. El l\'exico proporciona el componente sem\'antico de cada palabra en un formalismo concreto, y el analizador sem\'antico lo procesa para obtener una representaci\'on del significado de la frase.

\pagebreak