

\section{Caso de estudio}

\subsection{Minería de texto}

Definimos minería de texto como  un proceso mediante el cual se buscan patrones destacados y nuevos conocimientos en un conjunto de textos, es decir, es el proceso encargado de descubrir conocimientos que no existen tal cual en ningún texto del conjunto seleccionado pero que destacan al relacionar el contenido de varios de ellos.

Existen dos etapas principales dentro de este proceso: preprocesamiento y descubrimiento. 

El preprocesamiento consiste en transformar el texto en algún tipo de estructura que nos facilite su análisis. Por otro lado en el descubrimiento se analiza la estructura ateriormente mencionada teniendo como objetivo descubrir patrones interesantes o conocimientos nuevos.

