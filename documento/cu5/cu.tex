% \IUref{IUAdmPS}{Administrar Planta de Selección}
% \IUref{IUModPS}{Modificar Planta de Selección}
% \IUref{IUEliPS}{Eliminar Planta de Selección}

% 


% Copie este bloque por cada caso de uso:
%-------------------------------------- COMIENZA descripción del caso de uso.

%\begin{UseCase}[archivo de imágen]{UCX}{Nombre del Caso de uso}{
	\begin{UseCase}{CU3}{Enviar mensaje}{El caso de uso permite al Contacto enviar mensajes de texto a trav\'es de un canal abierto de comunicaci\'on.
	}
		\UCitem{Versi\'on}{0.2}
		\UCitem{Actor}{Contacto}
		\UCitem{Prop\'osito}{Enviar y Almacenar los mensajes escritos enviados entre los usuarios del sistema de mensajer\'ia, para su an\'alisis en el desarrollo de los m\'odulos posteriores.}
		\UCitem{Resumen}{
		El caso de uso permite al usuario enviar mensajes atrav\'es del un canal de comunicacion abierto}
		\UCitem{Entradas}{Mensaje string  (obligatorio)}
		\UCitem{Salidas}{Historial de mensajes enviados y recibidos }
		\UCitem{Precondiciones}{El usuario debera tener un canal de comunicaci\'on abierto.El mensaje no puede estar vac\'io.}
		\UCitem{Postcondiciones}{Los mensajes enviados y recibidos podr\'an ser visualidaos por ambos usuarios que comparten el mismo canal}		
		\UCitem{Autor}{Anah\'i Ruiz D\'iaz}
	\end{UseCase}

	\begin{UCtrayectoria}{Principal}
	
		\UCpaso[\UCactor] El usuario escribe un mensaje dentro de una caja de textos.
		\UCpaso[\UCactor] El usuario da clic en el bot\'on \IUbutton{Enviar} para enviar el mensaje. 

		\UCpaso  El sistema env\'ia y muestra al otro Contacto el mensaje enviado por el usuario.  \Trayref{A}.
		\UCpaso[] Fin del flujo.
				
	\end{UCtrayectoria}
		
		\begin{UCtrayectoriaA}{A}{Mensaje no enviado}
			\UCpaso El sistema muestra una alerta que notifica al usuario que ocurri\'o un error y su mensaje no fue enviado.
			\UCpaso[] El usuario da clic en el bot\'on \IUbutton{Aceptar}.
			\UCpaso[] Fin del flujo.
		\end{UCtrayectoriaA}


%-------------------------------------- TERMINA descripción del caso de uso.