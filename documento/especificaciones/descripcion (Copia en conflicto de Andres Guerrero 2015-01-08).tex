\section{Descripci\'on de prototipos}

La tabla \ref{tablaprototipos} muestra el nombre de los prototipos y sus caracter\'isticas generales.



\begin{table}[h]

\begin{center}


\begin{tabular}{|l|p{45mm}|p{80mm}|}
\hline

N & Nombre & Descripcici\'on \\

\hline



P1 & Identificador de Niveles 1, 2, 3, 4 y 6 & \begin{itemize}
\item Identificar Frases de nivel 1, 2, 3, 4 y 6.
\item Generar Vector de niveles.
\item Almacenamiento de vectores.
\end{itemize} \\

\hline 

P2 & Generador de Vectores de palabras Nivel 5 & \begin{itemize}
\item Estracci\'on de caracter\'isticas de nivel 5.
\item Generar vectores de frecuencias de nivel 5.
\item Almacenamiento de vectores. 

\end{itemize} \\

\hline 


P3 & Clasificador de conversaciones con incidencia en Nivel 5 & \begin{itemize}
\item Implementaci\'on de algoritmos de clasificaci\'on.
\item Entrenamiento de clasificador.
\item Pruebas del clasificador.
\end{itemize} \\

\hline 



P4 & Clasificador de comportamiento de conversaciones & \begin{itemize}
\item An\'alisis de vector de incidencias de Nivel 1, 2, 3, 4 y 6.
\item Clasificador: Red Neuronal.
\end{itemize} \\

\hline 

P5 & API de An\'alisis & \begin{itemize}
\item Integraci\'on de clasificadores.
\item M\'odulo de desici\'on.
\end{itemize} \\

\hline 

P6 & Sistema de Mensajer\'ia & \begin{itemize}
\item Sistema que simula conversaciones.
\item Sistema de pruebas fuera de l\'inea
\item Integraci\'on con la API de an\'alisis. 
\end{itemize} \\

\hline 


\end{tabular}
\caption{Prototipos y Funcionalidades Generales}
\label{tablaprototipos}

\end{center}

\end{table}