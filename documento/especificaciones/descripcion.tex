\section{Descripción}

Una vez definidos lo que son los IDS procederemos a los requisitos funcionales de cada componente del proyecto. \\

\subsection{Descripción Agente IDS}

% Requisitos funcionales \\

\begin{itemize}
    \item Se debe poder especificar los puertos de lectura de paquetes en el servidor.
    \item Los parametros de configuración del sensor deben estar ubicados en un archivo de configuración \textit{.conf}
    \item Solo se interceptaran y monitorearán paquetes HTTP/HTTPS.
    \item Para el caso del protocólo HTTPS el sistema deberá poder decodificiar el paquete usando las llaves generadas por el servidor.
    \item Implementar un algoritmo de \textit{machine learning} para identificar y asignar un rating de peligrosidad.
    \item Generar archivos logs con la fecha del evento, conexión o ID de sesión, conexión TCP, tipo de alerta, rating, protocolo, IP fuente y destino, puertos fuente y destino, cantidad de bytes transferidos, información del paquete decodificada.
    \item Utilizar concurrencia para ir liberando la carga de lectura de paquetes y aumentar la velocidad del agente.
\end{itemize}

% Requisitos no funcionales \\

% \begin{itemize}
%     \item Se debe poder especificar los puertos de lectura de paquetes en el servidor.
% \end{itemize}

\subsection{Descripción registrador de eventos}

% Requisitos funcionales \\

\begin{itemize}
    \item Se debe poder especificar la ruta donde se generarán los logs que vienen desde el agente del IDS.
    \item Se debe poder especificar la conexión a base de datos en la que se guardarán los eventos obtenidos de los logs.
    \item Se debe poder especificar la conexión a un servicio de caché que permita conectarse a un servidor de websockets para enviar notificaciones.
    \item Se debe poder especificar la conexión a un servicio de mail que permita enviar correos a los administradores del IDS.
    \item Los parametros de configuración del registrador de eventos tales como conexión a base de datos, url de los logs, etc. deben estar ubicados en un archivo de configuración \textit{.conf}
    \item Leer archivos logs que provengan del agente IDS.
    \item Analizar estructura de logs coincida con la estructura de un evento.
    \item Guardar eventos obtenidos desde los logs en una base de datos local o remota.
    \item Guardar eventos obtenidos desde los logs en una servicio de caché local o remoto.
    \item Enviar mail de acuerdo a las configuraciones por administrador de la consola para cada evento.
    \item El envío de mails debe de implementar concurrencia para evitar la carga de recursos al sistema.
\end{itemize}

% Requisitos no funcionales \\

% \begin{itemize}
%
% \end{itemize}

\subsection{Descripción consola de administración}

% Requisitos funcionales \\

\begin{itemize}
    \item Se deberán poder administrar usuarios para el sistema.
    \item Los usuarios del sistema pueden iniciar sesión.
    \item La contraseña de los usuarios tiene que estar encriptada con un algoritmo sha256.
    \item Deberán existir permisos para la realización de actividades dentro del sistema.
    \item Deberá existir siempre forzozamente un usuario de tipo superuser encargado de administrar todo el sistema, es decir contará con todos los permisos del sistema.
    \item Puede existir más de un superuser.
    \item Los usuarios superusers, podrán dar de alta otros usuarios superusers.
    \item Existe de la posibilidad de que haya usuarios tipo administrador con permisos especificos dentro del sistema.
    \item Los usuarios superusers, podrán dar de alta, modificar y eliminar a los usuarios adiministradores y administrar sus permisos.
    \item Los usuarios no se eliminan en su totalidad, solo se desactivan y se almacena la fecha que se desactivó.
    \item Los usuarios superusers, podrán eliminar otros usuarios superuser pero no a él mismo.
    \item Los usuarios podrán cambiar su contraseña en cualquier momento.
    \item Deberán existir grupos para así asignar permisos a un grupo y relacionar los usuarios con ese grupo.
    \item Deberán poderse administrar los permisos de los grupos.
    \item El sistema debe ser capaz de conectarse a la base de datos del IDS para obtener los eventos registrados.
    \item Deberá existir un permiso para ver los eventos del sistema.
    \item No se pueden eliminar eventos de la base de datos del IDS
    \item Se puede cambiar el estatus de los eventos en la base de datos del IDS y asignar una acción preventiva al evento.
    \item Todas las acciones de los usuarios se deben almacenar con la fecha, la acción y un mensaje del porqué se realizó ese cambio.
    \item Deberá existir un permiso para recibir notificaciones.
    \item Los usuarios superuser siempre tendrán permisos para recibir notificaciones.
    \item Si se tiene permiso para recibir notificaciones se deben poder administrar estás notificaciones según el tipo.
    \item Deberá existir un permiso para generar reportes.
    \item Los usuarios superuser siempre tendrán permisos para generar reportes.
    \item Los reportes se deben poder descargar en formato PDF.
    \item Deberá exsitir una vista con el índice de todos los eventos generados por el IDS y se deberá poder filtrar por ranking, status, protocólo y fecha.
    \item Es necesario que exista un monitoreo de eventos en tiempo real.
\end{itemize}

% Requisitos no funcionales \\

% \begin{itemize}
%
% \end{itemize}
