%---------------------------------------------------------
\section{Procesos}
\begin{description}
\item Flujo del proceso, iniciando desde el primer contacto con la empresa, hasta el pago de los servicios otorgados por la misma empresa. Cabe mencionar que durante este proceso pueden intervenir varias personas que mencionaremos posteriormente.
\item 1. El proceso inicia con el contacto con la empresa, este contacto lo pueden realizar clientes externos o aseguradoras.
Los primeros generalmente para cotizar precios y/o tiempos y realizar una contratacion, los segundos para solicitar un servicio por contrato entonces se registran los datos del cliente: nombre, automovil, ubicacion del automovil, tipo de servicio requerido y unidades requeridas. en este punto se puede determinar el destino o hasta que la grua llega al vehiculo depende si es externo o aseguradora.
\item 2. Paso siguiente el cliente decide si requiere de los servicios de la empresa.
\item 3. Cuando si es requerida, la empresa contacta al operador asignado para realizar el servicio con la unidad adecuada.
\item 4. A continuacion se envia la grua al lugar donde se ubica el vehiculo que se va a trasladar.
\item 5. Posteriormente el operador se comunica con el usuario.
\item 6. Se realiza el servicio (puede ser mas de un tipo de servicio dependiendo la situacion).
\item 7. Se determina hacia donde van a trasladar el automovil, puede trasladarse a un domicilio particular o puede llevarse a un taller de carros si es que la aseguradora asi lo determina.
\item 8. Finalmente se realiza el pago, este puede ser efectivo, a credito, de contado, deposito bancario convenido con el cliente o la aseguradora.
\item A grandes rasgos son todos los pasos que se realizan al brindar un servicio de gruas. Es importante mencionar que en este proceso intervienen varias y diferentes personas en cada servicio, debido a que las situaciones no son las mismas. \item Entre estas personas se encuentran: 
\item Personal administrativo 
 \item Operadores 
\item Clientes externos o aseguradoras
\item  Pueden intervenir o no: 
\item Autoridades estatales 
\item  Autoridades municipales 
\item  Autoridades federales 
 \item Ajustadores 
\item Peritos 
\item  Auxiliar de apoyo
\item Estas personas tambien tienen influencia sobre el desempeno del servicio, debido a que algunas veces el flujo del proceso del servicio de gruas se ve obstruido o acelerado por esta intervencion, dependiendo a cada caso. Con esto se quiere decir que no en todas las ocasiones, el proceso que lleva a cabo Gruas y Transportes Vazquez es independiente.
\item REGISTRO DE ORDEN
\item 1.- El personal administrativo en turno recibe la llamada del usuario y registra en una libreta los siguientes datos:
\item Numero Celular, nextel o telefonico de contacto: Nombre del Cliente: Empresa O Particular: IDcontrato:
\item El Conductor puede ser distinto al cliente: Datos del Vehiculo Marca: Tipo:
\item Color: Placas: Ubicacion de Origen Entre calles: Delegacion:
\item 2.- Colonia: calle donde esta el vehiculo: esquina con: Referencias Numero visible frente al auto: Estableciento o banco visible: otras: Destino: Delegacion: Colonia: calle de destino:
\item Tarifa: Subarrendamiento de filiales si/no Servicio gratuito o no Costo:
\item 3.- Radia a los operadores mas cercanos para saber quien esta disponible o se desocupa pronto independiente de su ubicacion, registrando: Numero de Grua(s) que se envia: Nombre del operador:
Ubicacion: Tiempo aproximado de llegada al lugar:
\item 4.- Espera la comunicacion del operador para registrar:
\item Reporte de confirmacion de llegada al lugar de origen: Reporte de confirmacion de llegada al lugar de destino: Reporte de Cobro.





\item El operador debe registrar los datos del beneficiario del servicio.
\item El operador debe registrar la placa, marca, submarca, modelo, capacidad de arrastre, capacidad de vehiculos, tipo de grua, conductores, corrales y clientes  de los vehiculos.
\item El operador debe llevar un registro de los servicios otorgados a los clientes, de tal forma que pueda conocer el nombre del cliente, fecha y hora, origen y destino del arrastre, vehiculo utilizado para el arrastre, datos del vehiculo arrastrado, conductor encargado del servicio, monto cobrado por el servicio, tipo de servicio, forma en que fue solicitado el servicio, persona que atendio la solicitud y la duracion del servicio.
\item El operador debe registrar la hora a la que inicia y a la que termina el servicio.
\item El operador requiere saber si el cliente esta registrado o no 
\item En el caso en el que no este registrado, el operador debe registrar un nuevo cliente.
\item El operador requiere conocer cual de las gruas esta mas proxima al lugar donde se llevara acabo el servicio.
\item El operador debe identificar a que tipo de cliente se le brindara el servicio.
\item En el caso en el que el tipo de servicio sea solicitado por una organizacion, el operador debe conocer si el convenio es vigente.
\item El operador requiere conocer el monto que se cobrara por el servicio prestado.
\item El operador requiere conocer la ubicacion de las gruas para determinar cual esta mas cercana al lugar del arrastre.
\item El conductor requiere conocer la direccion a la cual se tiene que dirigir para realizar el servicio, el tipo de servicio, y el monto por cobrar.



\end{description}
%---------------------------------------------------------
\section{Propuesta}

\begin{description}



\item Problematica
\item A continuacion se analizara la problematica de la empresa y se propondra una solucion.
Con base a la investigacion de los procesos de la empresa, se detectaron puntos no atendidos que interrumpen el optimo funcionamiento del servicio.
\item 1. El sistema actual no proporciona un medio para escoger a la grua mas cercana al cliente que necesita el servicio cuando el personal administrativo levanta una orden.
\item 2. Cuando se da un servicio a la aseguradora y todas la gruas se encuentra fuera de base, no hay forma de saber su localizacion exacta, por lo tanto no se sabe cual grua puede proporcionar el proximo servicio, para que este se realice de una forma mas eficiente.
\item 3. No cuentan con un sistema de registro eficiente. El registro de los servicios se hace manualmente
\item 4. No hay forma de saber si el operador de la grua proporciona un servicio externo del cual no se haya levantado una orden.
\item	Propuesta
\item Con base en el analisis de los procesos y los puntos no atendidos que hacen que los procesos sean ineficientes proponemos la siguiente solucion:
\item Tecnologia
\item Se tiene planteado realizar un sistema web, el cual no necesita de instalacion en cada una de las computadoras que se tenga dentro de la empresa. Se propone el desarrollo de un sistema diferente al que cuenta, siendo este mas automatizado, agilizando el proceso y ayudando a brindar un mejor servicio al cliente.


\end{description}


%---------------------------------------------------------
\section{Consecuencias estimadas}

\begin{description}



\item De continuar con el antiguo sistema para el manejo de gruas, podemos esperar lo siguiente: 
\item No se optimiza el uso de las gruas, ya que no se garantiza la seleccion de la primer grua disponible y de la mas cercana al servicio solicitado, lo cual impide se alcance el maximo beneficio por el uso de estas. 
\item No se minimiza el tiempo de espera del cliente que necesita el servicio, pudiendo ocasionar la cancelacion del mismo y ocasionando perdidas para el negocio. 
\item Hay huecos en el sistema actual que permiten el uso fraudulento y abusivo de las gruas por parte de los operadores, porque no se garantiza que todos os servicios sea dados de alta en el registro, y por ende no se garantiza el pago de estos. \item La contabilidad requiere m?as tiempo, debido al registro manual, y no es totalmente confiable por cuestiones como no garantizar el registro y cobro de todos los servicios proporcionados.


\end{description}


%---------------------------------------------------------
\section{Requerimientos de usuario}
\begin{tabular}{||l | l | r||} \hline \hline Req & Descripcion  \\ \hline
RU1 & El operador debe registrar los datos del beneficiario del servicio. \\ \hline 
RU2 & El operador debe registrar la placa, marca, submarca, modelo, capacidad de arrastre, capacidad de vehiculos, \\ & tipo de grua, conductores, corrales y clientes  de los vehiculos.
 \\ \hline 
RU3 & El operador debe llevar un registro de los servicios otorgados a los clientes, de tal forma que pueda conocer \\ &el nombre del cliente, fecha y hora, origen y destino del arrastre, vehiculo utilizado para el arrastre, datos del \\ &vehiculo arrastrado, conductor encargado del servicio, monto cobrado por el servicio, tipo de servicio, forma en  \\ & que fue solicitado el servicio, persona que atendio la solicitud y la duracion del servicio. \\ \hline 
RU4 & El operador debe registrar la hora a la que inicia y a la que termina el servicio. \\ \hline 
RU5 & El operador requiere saber si el cliente esta registrado o no. \\ \hline 
RU6 & En el caso en el que no este registrado, el operador debe registrar un nuevo cliente. \\ \hline 
RU7 & El operador requiere conocer cual de las gruas esta mas proxima al lugar donde se llevara acabo el servicio. \\ \hline 
RU8 & El operador debe identificar a que tipo de cliente se le brindara el servicio. \\ \hline 
RU9 & En el caso en el que el tipo de servicio sea solicitado por una organizacion, el operador debe conocer si el convenio \\ & es vigente. \\ \hline 
RU10 & El operador requiere conocer el monto que se cobrara por el servicio prestado.\\ \hline 
RU11 & El operador requiere conocer la ubicacion de las gruas para determinar cual esta mas cercana al lugar del arrastre. \\ \hline 
RU12 & El conductor requiere conocer la direccion a la cual se tiene que dirigir para realizar el servicio, \\ & el tipo de servicio, y el monto por cobrar. \\ \hline 

\end{tabular}

%---------------------------------------------------------
\section{Requerimientos de sistema}
\begin{description}

\item REQUERIMIENTOS FUNCIONALES

\begin{tabular}{||l | l | r||} \hline \hline Req & Descripcion  \\ \hline
RSF1 & El sistema validara si el cliente ya existe dentro de la base de datos. \\ \hline 
RSF2 & El sistema mostrara un formulario por medio del cual se pueda registrar los datos del beneficiario \\ & del servicio, es decir, el cliente, si este no esta registrado previamente. \\ \hline 
RSF3 & El sistema solicitara que el operador introduzca informacion de los vehiculos mediante un formulario \\ & que contenga los datos placa, marca, submarca, modelo, capacidad de arrastre, capacidad de \\ & vehiculos, tipo de grua, conductores, corrales y clientes.\\ \hline 
RSF4 & El sistema solicitara que el operador introduzca informacion de los servicios otorgados a los clientes \\ & mediante un formulario que contenga los datos nombre del cliente, fecha y hora, origen y destino del \\ & arrastre, vehiculo utilizado para el arrastre, datos del vehiculo arrastrado, conductor encargado del \\ &servicio, monto  cobrado por el servicio, tipo de servicio, forma en que fue solicitado el servicio, persona\\ & que atendio la solicitud y  la duracion del servicio.
 \\ \hline 
RSF5 & El sistema verificara el tipo de cliente para saber que tipo de servicio se le otorgara a este. \\ \hline 
RSF6 & El sistema verificara y notificara si el convenio esta vigente, en caso de que el servicio sea \\ & solicitado por alguna organizacion. \\ \hline 
RSF7 & En el caso en el que el servicio se le otorgue a un socio, el sistema calculara dependiendo \\ & del tipo de membresia, tipo de cliente o tipo de servicio, el monto por cobrar y en caso de tener arrastre\\ & de cortesia, se notificara. \\ \hline 
RSF8 & El sistema solicitara mediante un formulario que el operador ingrese las ubicaciones de los\\ & vehiculos. \\ \hline 
RSF9 & El sistema calculara la distancia entre las gruas disponibles y el lugar donde se llevara acabo \\ &el servicio para asi notificar al operador mediante una alerta cual es la grua mas cercana al lugar, en \\ &caso de no haber alguna disponible se notificara al usuario. \\ \hline 
RSF10 & El sistema calculara el tiempo aproximado que tardara en llegar el vehiculo al lugar de arrastre.\\ \hline 
RSF11 & El sistema solicitara mediante un formulario que el operador introduzca la hora a la que \\ &llega el vehiculo al lugar del servicio, asi como la hora real de terminacion del servicio. \\ \hline 
RSF12 & El sistema solicitara al operador la ubicacion en la que se encuentra el cliente. \\ \hline 
RSF13 & El sistema permitira al conductor consultar la ubicacion a la que se tiene que dirigir para\\ & realizar el servicio, el tipo de servicio y el monto por cobrar. \\ \hline  
\end{tabular}



\item REQUERIMIENTOS NO FUNCIONALES

\begin{tabular}{||l | l | r||} \hline \hline Req & Descripcion  \\ \hline
RSNF1 & El sistema sera escalable, es decir, ser podran hacer incrementos de este en caso de que \\ & fuera necesario. \\ \hline 
\end{tabular}





\end{description}
