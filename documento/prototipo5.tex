\section{An\'alisis}

\subsection{Descripci\'on}
Este prototipo se encarga de la toma de desiciones del clasificador de nivel 5 y el clasificador de niveles  1, 2, 3, 4 y 6.

\subsection{Objetivo}
Desarrollar el API de desici\'on a partir de la uni\'on de los clasificadores de nivel 5 y niveles 1, 2, 3, 4 y 6.

\subsection{Caracter\'isticas}
\begin{description}
\item[FEAT1:] La API recibe como entrada conversaciones en archivos de texto plano.
\item[FEAT2:] La API recibe como entrada conversaciones en archivos con formato xml.
\item[FEAT4:] La API preprocesa textos.
\item[FEAT4:] La API genera vectores de nivel 5.
\item[FEAT5:] La API genera vectores de nivel  1, 2, 3, 4 y 6.
\item[FEAT6:] El API dar\'a como salida la desici\'on del clasificador dle prototipo4.

\end{description}

\section{Dise\~no}


\section{Arquitectura}

La figura \ref{fig:arquitecturap5} muestra la arquitectura de la API de an\'alisis.

\begin{figure}[h]
\begin{center}
\includegraphics[scale=.3]{images/api}
\caption{Arquitectura de la API de ana\'alisis}
\label{fig:arquitecturap5}
\end{center}
\end{figure}


\subsection{Diagrama de clases}


La figura \ref{fig:dclasesp5} muestra el diagrama de clases de la API de an\'alisis.
\begin{figure}[h]
\begin{center}
\includegraphics[scale=.5]{images/clasesnivel5}
\caption{Diagrama de clases de la API de ana\'alisis}
\label{fig:dclasesp5}
\end{center}
\end{figure}

\section{Pruebas}


La tabla  \ref{tab:resultadasapi} muestra la tabla de resultados de conversaciones nuevas y su respectivo resultado.

\begin{table}
\begin{center}


\begin{tabular}{|l|c|c|c|c|c|c|c|c|}

\hline
Conversaci\'on & N1 & N2 & N3 & N\_3 & N4 & N5 & N6 & Decisi\'on \\
\hline
pruebas/pr10.txt &
 0 &
 3 &
 0 &
 0 &
 0 &
 1 &
 1 &
Muy Peligrosa \\
pruebas/pr11.txt &
 0 &
 4 &
 0 &
 0 &
 0 &
 1 &
 0 &
Muy Peligrosa \\ 
pruebas/pr12.txt &
 0 &
 0 &
 6 &
 4 &
 0 &
 1 &
 0 &
Muy Peligrosa \\ 
pruebas/pr13np.txt &
 0 &
 1 &
 0 &
  0 &
 0 &
 0 &
 0 &
No Peligrosa \\ 
pruebas/pr14.txt &
 0 &
 2 &
 0 &
 0 &
 0 &
 0 &
 0 &
No Peligrosa \\ 
pruebas/pr15.txt &
 0 &
 13 &
 4 &
 2 &
 1 &
 0 &
 0 &
Peligrosa \\
pruebas/pr16.txt &
 1 &
 4 &
 1 &
 0 &
 0 &
 1 &
 0 &
Muy Peligrosa \\ 
pruebas/pr17.txt &
 1 &
 0 &
 0 &
 0 &
 1 &
 1 &
 0 &
Muy Peligrosa \\ 
pruebas/pr18.txt &
 4 &
 7 &
 5 &
 3 &
 0 &
 1 &
 0 &
Muy Peligrosa \\ 
pruebas/pr19np.txt &
 0 &
 2 &
 0 &
 0 &
 0 &
 0 &
 0 &
No Peligrosa \\ 
pruebas/pr1.txt &
 2 &
 10 &
 0 &
 0 &
 1 &
 1 &
 3 &
Muy Peligrosa \\ 
pruebas/pr20np.txt &
 0 &
 2 &
 0 & 
 0 &
 0 &
 0 &
 0 &
No Peligrosa \\ 
pruebas/pr21.txt &
 1 &
 1 &
 0 &
 0 &
 0 &
 1 &
 0 &
Muy Peligrosa \\ 
pruebas/pr22.txt &
 0 &
 4 &
 0 &
 0 &
 0 & 
 1 &
 0 &
Muy Peligrosa \\ 
pruebas/pr2np.txt &
 0 &
 1 & 
 1 &
 0 &
 0 &
 0 &
 0 &
No Peligrosa \\
pruebas/pr3np.txt &
 1 &
 2 &
 0 &
 0 &
 0 &
 0 &
 0 &
No Peligrosa \\
pruebas/pr4.txt &
 0 &
 1 &
 0 &
 0 &
 0 &
 1 &
 0 &
Muy Peligrosa \\
pruebas/pr5.txt &
 0 &
 0 &
 0 &
 0 &
 0 &
 1 &
 0 &
Muy Peligrosa \\
pruebas/pr6.txt &
 0 &
 1 &
 3 &
 1 &
 0 &
 1 &
 0 &
Muy Peligrosa \\ 
pruebas/pr7.txt &
 1 &
 6 &
 4 &
 2 &
 0 &
 1 &
 0 &
Muy Peligrosa \\ 
pruebas/pr8.txt &
 0 &
 1 &
 1 &
 0 &
 1 &
 1 &
 0 &
Muy Peligrosa \\
pruebas/pr9.txt &
 0 &
 0 &
 0 &
 0 &
 0 &
 1 &
 0 &
Muy Peligrosa \\ 
pruebas/pr23.txt &
 1 &
 2 &
 0 &
 1 &
 0 &
 0 &
 0 &
No Peligrosa \\ 
pruebas/pr24.txt &
 0 &
 1 &
 0 &
 0 &
 0 &
 1 &
 1 &
No Peligrosa \\ 
pruebas/pr25.txt &
 0 &
 0 &
 0 &
 0 &
 0 &
 1 &
 0 &
Muy Peligrosa \\ 
\hline
\end{tabular}

\caption{Tabla de resultados de la Api}
\label{tab:resultadasapi}
\end{center}
\end{table}

En la figura \ref{fig:ClasificacionNiv} podemos observar el resultado de las pruebas realizadas en 25 conversaciones distintas. Los resultados muestran el n\'umero de incidencias de las frases categorizadas anteriormente en los niveles 1, 2 ,3, 4 y 6; as\'i como el n\'umero de conversaciones que conten\'ian palagras de car\'acter sexual.

\begin{figure}[h]
\begin{center}
\includegraphics[scale=.3]{images/grafica/ClasificacionNiv}
\caption{Resultados del muestro de 25 conversaciones}
\label{fig:ClasificacionNiv}
\end{center}
\end{figure}


En la tabla \ref{tab:resultadosM} se muestran los resultados que obtuvimos de las 25 conversaciones anteriormente mencionadas. La tabla contiene el n\'umero de conversaciones que se encontraron de las 4 diferentes categor\'ias: No peligrosa, Poco peligrosa, Peligrosa, Muy peligrosa.

\begin{table}
\begin{center}


\begin{tabular}
\hline
Clasificac\'on & Incidencias \\
\hline

No Peligrosa & 7\\
Poco Peligrosa & 0\\
Peligrosa & 1\\
Muy peligrosa & 17\\
\hline
\end{tabular}

\caption{Resultados de muestreo de 25 conversaciones}
\label{tab:resultadosM}
\end{center}
\end{table}

En la figura \ref{fig:ClasificacionConv} podemos observar graficamente la tabla anterior. Los resultados del muestreo de 25 conversaciones muestran que la mayor\'ia de \'estas (17) fueron detectadas como peligrosas.

\begin{figure}[h]
\begin{center}
\includegraphics[scale=.3]{images/grafica/ClasificacionConv}
\caption{Resultados del muestro de 25 conversaciones}
\label{fig:ClasificacionConv}
\end{center}
\end{figure}


