
\begin{thebibliography}{99}


\bibitem{uno} Kaplan J., Sharma S. \& Weinberg A. (2011). "Meeting the cybersecurity challenge". McKinsey \& Company. Recuperado 12 Marzo 2017, de \url{http://www.mckinsey.com/business-functions/business-technology/our-insights/meeting-the-cybersecurity-challenge}

\bibitem{dos} THE EDITORS AT CYBERSECURITY VENTURES. (2014). "The Cybersecurity Market Report covers the business of cybersecurity, including market sizing and industry forecasts, spending, notable M\&A and IPO activity, and more..'' Cybersecurity Ventures. Recuperado 26 Septiembre 2016, de \url{http://cybersecurityventures.com/cybersecurity-market-report/}

\bibitem{tres} King, S. (2016). ''Assessing the real risk of being online". ComputerWeekly. Recuperado 26 Septiembre 2016, de http://\url{www.computerweekly.com/feature/Assessing-the-real-risk-of-being-online}

\bibitem{cuatro} Computer Hope (2016). "Why do people hack computers?". Computerhope.com. Recuperado 26 Septiembre 2016, de \url{http://www.computerhope.com/issues/ch001530.htm}

\bibitem{cinco} Cloudbric. (2016). "6 Reasons Why Hackers Want to Hack Your Website". Recuperado 26 Septiembre 2016, de \url{https://www.cloudbric.com/blog/2015/10/6-reasons-why-hackers-want-to-hack-your-website/}

\bibitem{seis} Imperva Inc. (2016). "2015 Web Application Attack Report (WAAR)". WAAR 2015. Recuperado de \url{https://www.imperva.com/docs/HII\_Web\_Application\_Attack\_Report\_Ed6.pdf}

\bibitem{cookie} Gutierrez, E. (2009). "JavaScript". 1st ed. Barcelona: Ed. ENI, p.233.

\bibitem{siete} Assis, R. (2016). "Primero post de la serie sobre vulnerabilidades XSS". Sucuri Español. Recuperado 19 Diciembre 2016, de \url{https://blog.sucuri.net/espanol/2016/04/pregunte-sucuri-que-es-una-vulnerabilidad-xss.html}

\bibitem{ocho} (2016). "Qué es y cómo funciona un ataque Cross - Site Scripting". Hostalia. Recuperado 19 Diciembre 2016, de \url{http://pressroom.hostalia.com/wp-content/themes/hostalia\_pressroom/images/cross-site-scripting-wp-hostalia.pdf}

\bibitem{nueve} Ramos Pereira, K. (2016). ''Cross-Site Scripting". Revistasbolivianas.org.bo. Recuperado 20 Noviembre 2016, de \url{http://www.revistasbolivianas.org.bo/scielo.php?pid=S1997-40442013000100023\&script=sci\_arttext}

\bibitem{diez} (2014). "Su navegador esta desnudo: por qué los navegadores protegidos siguen siendo vulnerables.". Panda Security. Recuperado 27 Noviembre 2016, de \url{http://resources.pandasecurity.com/enterprise/solutions/7.\%20WP\%20PCIP\%20ESP\%20Su\%20Navegador\%20esta\%20desnudo.pdf}

\bibitem{once} Greene, T. (2016). "8 cyber security technologies DHS is trying to commercialize". Network World. Recuperado 26 Septiembre 2016, de \url{http://www.networkworld.com/article/3056624/security/8-cyber-security-technologies-dhs-is-trying-to-commercialize.html}

\bibitem{doce} (2016). "School of Computer Science and Information Technology University of Nottingham". Firewalls, Intrusion Detection Systems and Anti-Virus Scanners (p. 57). NOTTINGHAM NG8 1BB, UK. Recuperado de \url{http://citeseerx.ist.psu.edu/viewdoc/download?doi=10.1.1.107.2262\&rep=rep1\&type=pdf}

\bibitem{trece} Mookhey , K. K., Nilesh, B. (2011). "Detection of SQL Injection and Cross-site Scripting Attacks | Symantec Connect".
Symantec.com. Recuperado 26 Septiembre 2016, de \url{http://www.symantec.com/connect/articles/detection-sql-injection-and-cross- site-scripting-attacks10}

\bibitem{catorce} Tim, K. (2016). "Strategies to Reduce False Positives and False Negatives in NIDS | Symantec Connect". Symantec.com.
Recuperado 26 Septiembre 2016, de \url{http://www.symantec.com/connect/articles/strategies-reduce-false-positives-and-false-negatives-nids}

\bibitem{quince} Garcia-Alfaro, J. \& Navarro-Arribas, G. (2005). "Prevención de ataques de Cross-Site Scripting en aplicaciones Web".Recuperado 25 Noviembre 2016, de \url{http://www-public.tem-tsp.eu/~garcia\_a/web/papers/recsi08-xss.pdf}

\bibitem{dieciseis} González Márquez, V. (2009). "Sistema de detección de intrusos basado en sistema experto (Tesis de maestría)". Centro de Investigación en Computación. México.

\bibitem{proxy} National Institute of Standards and Technology,. (2007). "Guidelines on Securing Public Web Servers" (p. 121). Washington.

\bibitem{nist94} National Institute of Standards and Technology,. (2007). "Guide to Intrusion Detection and Prevention Systems (IDPS)" (p. 9). Washington.

\bibitem{nist41} National Institute of Standards and Technology,. (2009). "Guidelines on Firewalls and Firewall Policy" (p. 7). Washington.

\bibitem{nist33} National Institute of Standards and Technology,. (2001). ''Underlying Technical Models for Information Technology Security" (p. 6). Washington.

\bibitem{nist53} National Institute of Standards and Technology,. (2014). "Summary of NIST SP 800-53 Revision 4, Security and Privacy Controls for Federal Information Systems and Organizations" (p. 6). Washington.

\bibitem{CIDF} Tung, B. (1999). ''Common Intrusion Detection Framework". Gost.isi.edu. Recuperado 15 Abril 2017, de \url{http://gost.isi.edu/cidf/}

\bibitem{arquitecturas} Mira, J. (2017). ''Implantación de un Sistema de Detección de Intrusos en la Universidad de Valencia" (1ra ed., p. 15-21). Valencia: Recuperado de \url{http://rediris.es/cert/doc/pdf/ids-uv.pdf}

\bibitem{IEFT} Anónimo. (2017). ''Internet Engineering Task Force (IETF)". Ietf.org. Recuperado 16 Abril 2017, de \url{https://www.ietf.org/}

\bibitem{XML} Anónimo. ''Extensible Markup Language (XML)". (2017). W3.org. Recuperado 21 Abril 2017, de \url{https://www.w3.org/XML/}

\bibitem{defmach} Tsai, J., \& Yu, Z. (2011). ''Intrusion detection" (1st ed.). London: Imperial College Press.



\end{thebibliography}

