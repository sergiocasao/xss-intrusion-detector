\begin{titlepage}


%\begin{center}
%\begin{tabular}{r c}
%\includegraphics[scale=.3]{images/ipn.jpg} & \LARGE \textbf{Instituto Polit\'ecnico Nacional} & \includegraphics[scale=.3]{images/escom.jpg} \\
%& \Large \textbf{Escuela Superior de C\'omputo} &
%\end{tabular}
%\end{center}

\begin{center}
\begin{tabular}{r c l}
\includegraphics[scale=.10]{images/ipn.jpg} & \huge \textbf{INSTITUTO POLITÉCNICO NACIONAL} & \includegraphics[scale=.25]{images/escom.jpg}\\ 
& \Large \textbf{ESCUELA SUPERIOR DE CÓMPUTO}
\end{tabular}
\end{center}


\vspace{1.5cm}
\begin{center}
\large Trabajo Terminal: \linebreak

\large \textbf{``Sistema de Detección de Intrusos para ataques Cross-Site Scripting''} \linebreak
\large 2016-B089

\end{center}

\vspace{1.5cm}

\begin{center}
Presentan: \linebreak
\textbf{Fonseca Casao Sergio Israel} \linebreak
\textbf{Israel García Ramírez} \linebreak
\end{center}

\vspace{1.5cm}

El presente documento se encuentran los temas relacionados al desarrollo del Trabajo Terminal I cuyo objetivo es el desarrollo de un sistema de detección de intrusos que se especifique en la detección de ataques Cross-Site Scripting (XSS) mediante la aplicación de aprendizaje máquina. Este sistema pretende servir como herramienta capaz de detectar y alertar a los administradores de los servidores de un ataque XSS en curso y almacenar los datos necesarios para su posterior análisis.\linebreak

\textbf{Palabras Clave}:  Aprendizaje Automático, Cross-Site Scripting, Lenguajes Regulares, Sistema de Detección de Intrusos.

\vspace{1.5cm}
 
\begin{center}


Directores: \linebreak
\textbf{ M. en C. Ramírez Morales Mario Augusto, M. en E. Saucedo Delgado Rafael Norman}

\end{center}





\end{titlepage}
