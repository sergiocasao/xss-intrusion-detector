\begin{titlepage}


%\begin{center}
%\begin{tabular}{r c}
%\includegraphics[scale=.3]{images/ipn.jpg} & \LARGE \textbf{Instituto Polit\'ecnico Nacional} & \includegraphics[scale=.3]{images/escom.jpg} \\
%& \Large \textbf{Escuela Superior de C\'omputo} &
%\end{tabular}
%\end{center}

\begin{center}
\begin{tabular}{r c l}
\includegraphics[scale=.25]{images/ipn.jpg} & \huge \textbf{INSTITUTO POLIT\'ECNICO NACIONAL} & \includegraphics[scale=.25]{images/escom.jpg}\\ 
& \Large \textbf{ESCUELA SUPERIOR DE C\'OMPUTO}
\end{tabular}
\end{center}


\vspace{1.5cm}
\begin{center}
\large Trabajo Terminal: \linebreak

\large \textbf{``Sistema de an\'alisis de texto mediante reconocimiento de patrones. Caso de estudio: \textit{Online grooming}''} \linebreak
\large 2013B-017

\end{center}

\vspace{1.5cm}

\begin{center}
Presentan: \linebreak
\textbf{Guerrero Madrigal Andr\'es Emanuel} \linebreak
\textbf{Hern\'andez Hern\'andez Santos} \linebreak
\textbf{Ruiz D\'iaz Anah\'i} \linebreak
\end{center}

\vspace{1.5cm}


En el presente documento se encuentran los resultados correspondientes al desarrollo del Trabajo Terminal cuyo objetivo es la implementaci\'on de un sistema que analice mensajes mediante  procesamiento de lenguaje natural y teoría de reconocimiento de patrones. Este sistema pretende servir como herramienta capaz de clasificar la intención de dichas  conversaciones como peligrosas o no peligrosas. \linebreak

\textbf{Palabras Clave}:  Reconocimiento de patrones,  procesamiento de lenguaje natural, inteligencia artificial, aprendizaje máquina, acoso sexual infantil en la web.

\vspace{1.5cm}
 
\begin{center}


Directores: \linebreak
\textbf{ M. en C. Ram\'irez Morales Mario Augusto, M. en E.  Silva S\'anchez Carlos}

\end{center}





\end{titlepage}
