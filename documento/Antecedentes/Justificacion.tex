%---------------------------------------------------------
\section{Justificaci\'on}

La llegada de internet abrió las puertas a grandes posibilidades de comunicaci\'on: redes sociales, foros, chats, etc\'etera. El t\'ermino \textit{grooming} hace referencia a las acciones que lleva acabo un adulto para establecer amistad con un menor por medio de internet, con el objetivo de obtener una satisfacci\'on sexual mediante la obtenci\'on de im\'agenes con contenido er\'otico o sexual del menor. A pesar de que estas situaciones tienen su origen dentro de la red, muy frecuentemente terminan en abuso f\'isico a menores o tr\'afico de pornograf\'ia infantil.

Los problemas que se presentan para atacar el grooming son principalmente: La inocencia de los menores, el anonimato en el que se mantienen los adultos implicados y la facilidad con la que se puede acceder a internet hoy en d\'ia.

La detecci\'on de una posible amenaza hacia un infante podr\'ia realizarse de manera manual, donde un padre de familia tenga acceso a todos los mensajes que se comparten en un canal. Sin embargo, esta detecci\'on es poco factible ya que el an\'alisis de los mensajes podr\'ia ser tardado dependiendo del volumen de la  informaci\'on.

Es por ello que este sistema pretende actuar como una herramienta capaz de automatizar el an\'alisis de dichas conversaciones y con ello hacer uso del sistema implementado.