%---------------------------------------------------------
\section{Justificación}

La gran mayoría de los sistemas desarrollados hoy en día enfocados a la detección de intrusos basados en red, no tienen un gran soporte ante los ataques XSS, de tal forma que pueden llegar a fallar teniendo falsos positivos o falsos negativos \cite{trece}\cite{catorce}, y las herramientas que lo tienen mejor implementado son adquiridas por empresas que puedan absorber el pago debido a su costo alto.\\

Para intentar solucionar tanto los ataques XSS persistentes como los no persistentes se sugiere implementar un sistema de filtrado y/o análisis, aunque estas soluciones pueden ser propuestos teóricamente como una tarea fácil, llevarlo a la práctica es mucho más complicado. Aunque la mayoría de ataques XSS conocidos están escritos en JavaScript e incrustados en documentos HTML, aunque también se pueden usar otras tecnologías como Java, Flash, ActiveX, etc., para efectuar los ataques, es por ello que es muy complicado la concepción de un proceso de filtrado y/o análisis genérico capaz de tratar el mal uso de dichos lenguajes.\\

La complejidad para ser detectados radica por una parte, en la utilización de \textit{proxies}\footnote{Es una aplicación que ''rompe" la conexión entre el cliente y el servidor \cite{proxy}.}  de filtrado, especialmente en la parte del servidor, que introduce limitaciones importantes referentes a la escalabilidad y rendimiento de aplicaciones Web. Por otra parte, los scripts maliciosos pueden estar incrustados en los documentos intercambiados de manera ofuscada (por ejemplo codificando el código malicioso en hexadecimal o métodos de codificación avanzados) para no ser detectado ante estos filtros y analizadores \cite{quince}.\\

Se considera este proyecto ya que será de ayuda a aquellas empresas y personas que deseen detectar ataques de tipo XSS dirigidos a las aplicaciones instaladas en sus servidores, implementando métodos de análisis de datos, como el aprendizaje máquina orientados a la seguridad informática haciendo más eficiente su funcionamiento. Y así alertar a los administradores para poder prever efectos irreversibles en el sistema o de manera más grave, una toma de control total o escalabilidad de permisos en el sistema anfitrión del servicio.\\