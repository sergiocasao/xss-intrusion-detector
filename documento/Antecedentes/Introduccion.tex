%---------------------------------------------------------
\section{Introducción}

La seguridad informática ha sido y actualmente es un sector en el cual, empresas importantes de gran prestigio gastan cientos de
millones de dólares para protegerse al estar conectados a una red \cite{uno}, tal es la preocupación que a nivel mundial se registró una
inversión en la seguridad informática de 75 billones de dólares en 2015 \cite{dos}, mientras que, tanto chicas como medianas empresas
suelen gastan un mínimo relacionado a este tema. En la actualidad, con la era de la revolución tecnológica por la que se está
pasando, las empresas se han visto obligadas a contratar nueva tecnología para su producción, publicidad y/o servicios conectada al
mundo de la Internet. Las empresas al estar conectadas a la Internet, están conectadas a millones de usuarios con cientos de
posibilidades de acceso a los servicios de las empresas.\\

Cuando los usuarios se conectan a la red de Internet, están conectados todos los usuarios de la misma simultáneamente, esto
conlleva un alto riesgo de inseguridad. Para tratar de disminuir el impacto provocado por amenazas informáticas, existen
programas de computadora enfocados a detectar y proteger a los usuarios de la red contra los impactos provocados por las
amenazas \cite{tres}.\\

Hoy en día, existen diversos tipos de amenazas en la red, algunas son muy conocidas como virus informáticos que son diseñados
para infectar archivos, pero no sólo existen ese tipo de amenazas diseñadas cierta forma con una actuación “automatizada”.
También existen las amenazas humanas o conocidos como Piratas informáticos, los cuáles, por medio de diversas técnicas de
vulneración, pueden infectar, tomar el control o incluso obtener información o privilegios de computadoras o servidores con el fin
de aprender o poner en práctica nuevas técnicas de vulneración, vender la información obtenida en el mercado negro o inclusive
realizar un daño directo a los archivos o computadora objetivo \cite{cuatro}\cite{cinco}.\\

Con el incremento de nuevos servicios web en Internet, se han creado y desarrollado diversos tipos de ataques hacia los servidores
que proporcionan estos servicios. En los últimos años han ido en incremento los ataques web, aunque los que han tenido un mayor
crecimiento y un gran impacto a nivel global, son los ataques tipo Cross-Site Scripting (de ahora en adelante XSS) \cite{seis}.



\pagebreak