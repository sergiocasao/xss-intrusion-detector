%---------------------------------------------------------
\section{Descripci\'on del problema}
%En la actualidad se han desarrollado diversos sistemas que analizan textos  utilizando un conjunto de palabras filtro, si alguna palabra del texto cae dentro de ese grupo: se realiza una acci\'on. 

%Sin embargo una palabra puede ser utilizada en diversas situaciones y puede obtener diversos significados dependiendo de su contexto. Por lo cual herramientas como las descritas anteriormente pueden no ser tan certeras como se esperar\'ia. 

%El desarrollo de un sistema que no analice \'unicamente una lista de palabras sino que pueda realizar un analisis del contexto en el cual se encuentra la palabra puede mejorar los resultados obtenidos.

El ser humano a lo largo de su historia ha acumulado y sigue generando informaci\'on, de ah\'i que resulte complejo y se requiera de una gran cantidad de esfuerzo humano analizar, recuperar y clasficar informaci\'on.

% Por lo cual se necesita crear un sistema que realice el análisis de alguna conversación fuera de línea,  con esto nos referimos a que ha terminado el uso de una aplicación de intercambio de mensajes y  así poder clasificar el comportamiento de cada una de las fuentes (usuario que emitió el mensaje), utilizando reconocimiento de patrones. Una vez determinada la clasificación se podrá tomar una decisión.

%En particular el caso de estudio que trataremos es conocido como online grooming, que es definido como el proceso mediante el cual una persona se hace amigo de un niño en línea, con el  fin de realizar acoso y posteriormente abuso sexual.\cite{grooming}

%Actualmente surge el interés de crear sistemas que sirvan de apoyo al realizar el análisis de textos. Estos son algunos ejemplos de sistemas que se han desarrollado en diferentes partes del mundo.\\
